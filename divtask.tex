\appendix
\begin{center}
  \section{人员职责与分工}
\end{center}

\begin{itemize}
  \item 叶晓军:组长,主要负责项目人员组织、进度管理、系统业务逻辑设计
  \item 杨曦华:主要负责系统前端设计
  \item 柯毅豪:主要负责系统后台设计、系统部署
  \item 钟宇腾:主要负责系统架构设计、代码管理
  \item 系统测试与性能调优由全体成员共同完成。
\end{itemize}


\newpage

\begin{center}
  \section{项目开发使用的技术说明}
\end{center}

\subsection{系统架构}
\begin{itemize}
  \item Django Web~框架(项目主页:https://www.djangoproject.com/)
  
  \CJKindent 本系统基于~django web~框架进行开发。~Django Web~框架是一个高级~Python\footnotemark[1] Web~框架,提供~ORM (~Object-relational mapper)~方法,可以完全使用~Python~来定义数据库的表及其元素,并封装了大量的数据库访问~API~,把数据库元素映射为~Python~内部类型,兼容多种数据库后台;提供~Template~系统,可以在~HTML~中嵌入~Python~代码便于服务器动态生成页面;提供缓存模块,提高服务器运行效率。
  
  \item Bootstrap前端框架与交互组件集(项目主页:http://twitter.github.com/bootstrap/)
  
  \CJKindent 本系统前端基于~Bootstrap~模板进行开发。~Bootstrap~是用于快速开发~Web~应用的前端工具包。它是一个~HTML, CSS, JavaScript~的集合,它使用了最新的浏览器技术,给~Web~开发提供了时尚的版式、表单、按钮、表格、网格系统等等。
\end{itemize}

本系统是一个基于~Web 2.0~架构的系统,前端使用~AJAX (Asynchronous JavaScript and XML)~来动态显示内容和响应用户输入,服务器采用~Django~网络框架处理数据。在处理请求时不返回整个网页而是只返回用~JSON (Javascript Object Notation)~表示的数据,前端框架处理数据并呈现,不刷新整个页面而只更新页面的部分内容。

采用此架构可以节约网格资源,在服务器处理大量情求时(如选课时期),可以节约服务器资源和网络带宽,提高网站的响应速度,减轻服务器负担。

本系统在设计时充分考虑到跨平台的需要,兼容绝大多数主流的浏览器(包含桌面系统与手持设备系统的浏览器),在呈现时自动检测显示区域分辨率,若检测到是手持设备的分辨率,则会显示一个专门优化过的页面,提升用户体验。

\footnotetext[1]{Python~是一种面向对象的解析性编程语言、动态语言,支持命令式程序设计、面向对象程序设计、函数式编程、面向切面编程、泛型编程多种编程范式,具备垃圾回收功能,能够自动管理储存器使用。语言设计的哲学是“用一种方法,最好是只有一种方法来做一件事”,语法简洁少有岐义,代码可读性高。}

\subsection{代码托管与版本控制}
\begin{itemize}
  \item Git~版本控制系统
  
  \CJKindent Git~是一个由~Linus Benedict Torvalds~为了更好地管理~Linux~内核开发而创立的分布式版本控制/软件配置管理软件。与常用的版本控制工具~CVS、Subversion~等不同,它使用了分布式版本库,不需要服务器端软件支持,使源代码的发布和交流极其方便。~Git~的速度很快,这对于诸如~Linux kernel~这样的大项目来说自然很重要。~Git~最为出色的是它的合并跟踪(~merge tracing~)能力。
  
  \item GitHub
  
  \CJKindent GitHub~是一个基于互联网、使用~Git~版本控制系统的项目存取服务,提供了诸如~feeds~,~followers~,开发者的工作网络图表等统计服务。根据在2009年的~Git~用户调查,~GitHub~是最流行的~Git~存取站点。
  
  \CJKindent 我们的项目托管于~GitHub~,地址为~https://github.com/zonyitoo/SYSUEduAdminSystem~,使用~Git~版本控制系统来进行版本控制和软件配置管理。
\end{itemize}

\subsection{数据库}
我们的项目使用~PostgreSQL~数据库作为数据库后台,~PostgreSQL~是自由的对象-关系数据库服务器(数据库管理系统),在灵活的~BSD~许可证下发行。~PostgreSQL~使用~SQL~语言来在执行资料的查询。这些资料通过连外键联系在一起,以一系列表格的形式存在。~PostgreSQL~相对于竞争者的主要优势,主要的特征为可编程性:对于使用数据库资料的实际应用,~PostgreSQL~让开发与使用的工作,变得更加容易。与PostgreSQl配合的开源软件很多,有很多分布式集群软件,如pgpool、pgcluster、slony、plploxy等等,很容易做读写分离、负载均衡、数据水平拆分等方案。

教务系统经常需要处理高并发情形,~PostgreSQL~是多进程的,在并发不高时处理速度略慢于~MySQL~等多线程数据库,但当并发高的时候,对于现在多核的单台机器上,~PostgreSQL~能利用多核处理器的并发处理优势,整体性能优于其它基于多线程的数据库。

\newpage
\begin{center}
  \section{系统开发日志}
\end{center}

\begin{figure}[H]
   \centering \includegraphics[width=\textwidth]{img/contrib.png}
   \caption{项目成员贡献及项目提交数与时间关系图}
\end{figure}

上图中,帐号与成员的对应关系为:钟宇腾(~zonyitoo~)、杨曦华(~19thhell~)、叶晓军(~iphkwan~)、柯毅豪(~sheepke~)。我们的项目从2012年10月21日开始编写,至2012年12月23日完成。在2012年11月20日前后进行基础框架搭建,并推出~Alpha~测试版,在2012年12月16日前后对于系统细节修补及加入新功能,推出~Beta~测试版。

钟宇腾(~zonyitoo~)是项目代码的管理员,主要负责项目文件树管理及架构设计,因此提交数(~Commits~)最多,增加了60320行代码,删除78212行代码;杨曦华(~19thhell~)是网页前端开发者,主要负责网页前端设计,有115次提交,增加了54075行代码,删除9893行代码;叶晓军(~iphkwan~)是组长,主要负责项目进度管理,有46次提交,增加5256行代码,删除3042行代码;柯毅豪(~sheepke~)是系统后台开发者,主要负责后台服务器响应代码编写及服务器维护及测试,有17次提交,增加1009行代码,删除2072行代码。全组所有成员在基本分工下互相合作,并没有很绝对的分工界线。

\begin{figure}[H]
   \centering \includegraphics[width=\textwidth]{img/code_freq.png}
   \caption{项目代码变化图}
\end{figure}

图中虚线是代表代码行数的变化与时间的关系,2012年12月23日后代码行数超过了2.6万。中央横线以上的图中,高度代表该时间段增加的代码行数;中央横线以下的图中,高度代表该时间段被删除的代码行数,总体体现了代码增加的速度。\vspace{2em}

项目托管主页:https://github.com/zonyitoo/SYSUEduAdminSystem

项目统计图:https://github.com/zonyitoo/SYSUEduAdminSystem/graphs
